\documentclass[letterpaper]{resume}

% This is just to get better error messages for debugging:
\setcounter{errorcontextlines}{100}

% For more compact lists.
\usepackage{paralist}

\setlength{\textheight}{10in}
\setlength{\textwidth}{6.5in}

\setlength{\voffset}{-0.4in}
\setlength{\oddsidemargin}{0pt}

\begin{document}

%%%%%%%%%%%%%%%%%%%%%%%%%%%%%%%%%%%%%%%%%%%%%%%%%%%%%%%%%%%%%%%%%%%%%%%%%%%%%%%%%%%%%%%%%%%%%%%%%%%%%%%%%%%%%%%%%%%%%%%%%%%%%%%
% Header
%%%%%%%%%%%%%%%%%%%%%%%%%%%%%%%%%%%%%%%%%%%%%%%%%%%%%%%%%%%%%%%%%%%%%%%%%%%%%%%%%%%%%%%%%%%%%%%%%%%%%%%%%%%%%%%%%%%%%%%%%%%%%%%
\author{Sarah F. Majors}
%\email{sfmajors373@gmail.com}
\webpage{sfmajors373@gmail.com}
\phone{(724) 683 - 5500}
\streetaddress{https://www.linkedin.com/in/sarah-f-majors/}
\citystatezip{www.github.com/sfmajors373}
\maketitle

%%%%%%%%%%%%%%%%%%%%%%%%%%%%%%%%%%%%%%%%%%%%%%%%%%%%%%%%%%%%%%%%%%%%%%%%%%%%%%%%%%%%%%%%%%%%%%%%%%%%%%%%%%%%%%%%%%%%%%%%%%%%%%%
% Overview
%%%%%%%%%%%%%%%%%%%%%%%%%%%%%%%%%%%%%%%%%%%%%%%%%%%%%%%%%%%%%%%%%%%%%%%%%%%%%%%%%%%%%%%%%%%%%%%%%%%%%%%%%%%%%%%%%%%%%%%%%%%%%%%
\section{Overview}
\goodbreak\vspace{\secskip}\par\noindent\begin{tabularx}{\linewidth}{Xr}  
  Originally an archaeologist, I have transitioned to tech and have spent my first tech jobs working in web development with a side of data analysis and data visualization.  Now I am looking to obtain a job in artificial intelligence.
\end{tabularx}


%%%%%%%%%%%%%%%%%%%%%%%%%%%%%%%%%%%%%%%%%%%%%%%%%%%%%%%%%%%%%%%%%%%%%%%%%%%%%%%%%%%%
%%%%%%%%%%%%%%%%%%%%%%%%%%%%%%%%%%%%%%%%%%%%%%%%%%%%%%%%%%%%%%%%%%%%%%%%%%%%%%%%%%%
% Machine Learning Projects
%%%%%%%%%%%%%%%%%%%%%%%%%%%%%%%%%%%%%%%%%%%%%%%%%%%%%%%%%%%%%%%%%%%%%%%%%%%%%%%%%%%%%
%%%%%%%%%%%%%%%%%%%%%%%%%%%%%%%%%%%%%%%%%%%%%%%%%%%%%%%%%%%%%%%%%%%%%%%%%%%%%%%%%%%%%
\section{Machine Learning Projects}

% \affiliation[]{Lidar 3D Mapping Device}{In Progress}
% \begin{compactitem}
% \item Utilize a compact lidar module on a custom frame to collect detailed topographical information at close range
% \item Integrate data from individual units with the information provided by the Total Station to create site wide maps
% \item Create detailed, accurate, user friendly, and aesthetically pleasing maps
% \item Tech stack: QGIS, ROOT (Cern), Python, LaTeX
% \end{compactitem}
% 
% \affiliation[]{8-bit Computer}{In Progress}
% \begin{compactitem}
% \item Building a primitive 8-bit computer with breadboards and LED's based on Ben Eater's video series
% \item Planning PCBs to replace the breadboards for a more permanent, cat-proof arrangement
% \item Skills developed: Electrical Engineering, Computer Hardware Basics, Troubleshooting
% \end{compactitem}

% \affiliation[]{Machine Learning Projects}{In Progress}
% \begin{compactitem}
% \item Breast Cancer Capstone for FourthBrain
% \item Creating a demo of the application of privacy preserving methods to GIS in coordination with OpenMined for the United Nations
% \item Created a federated learning cluster on Raspberry Pi using ArchArm , PyTorch, and PySyft
% \item Analyzed the differences in Gaussian and Laplacian noise when used in differentiated learning on Android% using PySyft for Android
% \end{compactitem}

\affiliation[]{Breast Cancer Capstone}{}
\begin{compactitem}
\item Neural network was designed and trained against stained lymph node slides, from the Camelyon 16 dataset, to perform image segmentation with the goal of the detection of cancerous and abnormal cell structures
\item Various methods such as OTSU were used to efficiently clean, tile, and mask the slide data to reduce man hours necessary for production of a dataset as well as enhance the quality of the training set and model accuracy
\item Convolutional Neural Network model was designed to flag images from Camelyon 16 dataset as abnormal
\item UNet based image segmentation was performed agaisnt Camelyon 16 dataset images identified by the CNN
\item Pix2Pix Generative Adversarial Network was used to expand the existing Camelyon 16 dataset to create more training data to feed the pipeline to enhance reliablity and accuracy by balancing positive and negative samples
\item Both GPU and CPU computation were utilized to reduce training time and increase the velocity of development
\item Set up and deployed a Jupyter Labs based research environment in a Docker container based on Arch Linux on a 32 core Threadripper machine with a mirror of the Camelyon 16 dataset for each team member to reduce the set up time and allow for rapid iterations
\item Deployed the model in a flask application on Google Cloud allowing a user to upload a whole slide or just an image tile and see the prediction as well as the segmentation, if applicable
\end{compactitem}

\affiliation[]{Privacy Preserving Satellite Imagery}{in progress}
\begin{compactitem}
\item Produced a demo for Open Mined to implement a machine federated learning model in which data owners have control of the data and the data scientists never access the raw data thereby keeping the data private whilst still allowing the end user to have meaningful results
\item The model is trained on satellite imagery to locate pools in certain areas as a proof of concept that these methods of data privacy can be used on data of this nature
\item Created reusable Docker container hosting Jupyter Labs and all necessary machine learning libraries to reduce issues related to dependency management
\item Compiled PySyft and PyGrid packages for Arch Linux and became the maintainer of those and several other machine learning packages in the Arch User Repository
\item Built computer to reduce dependency on expensive cloud compute and prevent runaway costs while experimenting with various models, parameters, etc as well as speed up training time
% \item Worked on a demo intended for the United Nations that would utilize privacy preserving methods to pull data
% from satellite imagery without the end user every actually seeing the images
\end{compactitem}

\affiliation[]{Chest X-Ray Pathology Detection}{}
\begin{compactitem}
\item Utilized the CheXpert dataset to develop a potential low cost solution to detect chest pathologies, such as cardiomegaly, edema and pleural effusion, deployed to Raspberry Pi via Flask with three other people
\item Trained various models, such as VGG and Resnet, to achieve best results in given time, with VGG16 having the best results at 79.15\% accuracy rate with a 50,000 image training set
\item Created a small flask application to deploy on Raspberry Pi to upload images and make pathology predictions
\item Recorded video demo of the project for hackathon submission
\end{compactitem}


\affiliation[]{Federated Learning Cluster}{}
\begin{compactitem}
\item Created a federated learning computational cluster on a set of four Raspberry Pi 3s and 4s to have a platform to further explore how federated learning works
\item Built an automated tool to create Arch Linux ARM rootfs for Raspberry Pi 3s and 4s using pacman and bash
\item Practiced compiling Arch packages for ARM so I could use Arch on the Pis
\end{compactitem}
 
\affiliation[]{Noise Analysis on Android}{}
\begin{compactitem}
\item Worked with three others to compare accuracy of local and global differential privacy using federated learning via the PySyft library
\item Utilized Gaussian and Laplacian curves as the Lapace mechanism to analyze how the difference in noise impacted accuracy, in addition to testing without noise
\item Tested in an Android environment using the Synthetic Digits dataset and created outcome visualizations using GraphView
\end{compactitem}

% \affiliation[]{Cookbook}{In Progress}
% \begin{compactitem}
% \item Having tired of calling my mom to send me her recipes at midnight, I decided to compile them into a single space that was easily replicated and shared so my mom, my sister and I could all have a copy
% \item Used this as an opportunity to learn more LaTex while having fun and making something I actually need
% %\item Compile family recipes in LaTex cookbook
% \item Tech stack: LaTex, Git
% \end{compactitem}

%%%%%%%%%%%%%%%%%%%%%%%%%%%%%%%%%%%%%%%%%%%%%%%%%%%%%%%%%%%%%%%%%%%%%%%%%%%%%%%%%%%%%%%%%%%%%%%%%%%%%%%%%%%%%%%%%%%%%%%%%%%%%%%
% Continuing Education
%%%%%%%%%%%%%%%%%%%%%%%%%%%%%%%%%%%%%%%%%%%%%%%%%%%%%%%%%%%%%%%%%%%%%%%%%%%%%%%%%%%%%%%%%%%%%%%%%%%%%%%%%%%%%%%%%%%%%%%%%%%%%%%
% \section{Continuing Education}
% 
% \affiliation[]{Machine Learning Fundamentals}{In Progress}
% \begin{compactitem}
% \item Completed the AI in Medicine Specialization by deeplearning.ai
% \item Competing in Intel Edge AI scholarship challenge
% \item Competing in Bertelsman AI Deep Learning scholarship challenge
% \item Completed the Computer Vision Nano Degree from Udacity focusing on applying computer vision methods with the PyTorch and OpenCV libraries, projects include facial keypoint detection, image captioning and SLAM
% \item Completed and won the Secure and Private AI Facebook Scholarship Challenge Course focusing on differential and federated learning using PySyft and PyTorch with a heavy emphasis on group projects coordinated in the Slack community
% \item Completed and received certificate for the Deep Learning Specialization from deeplearning.ai covering deep neural networks, convolutional neural networks, sequence models and structuring machine learning projects
% \item Completed Machine Learning course by deeplearning.ai and Andrew Ng using Octave
% \item Completed the Bertelsman Data Science Challenge Scholarship Course focusing on statistics, python, and SQL
% \item Working to enhance my math understanding through MITx and Outlier
% \end{compactitem}

%%%%%%%%%%%%%%%%%%%%%%%%%%%%%%%%%%%%%%%%%%%%%%%%%%%%%%%%%%%%%%%%%%%%%%%%%%%%%%%%%%%%
%%%%%%%%%%%%%%%%%%%%%%%%%%%%%%%%%%%%%%%%%%%%%%%%%%%%%%%%%%%%%%%%%%%%%%%%%%%%%%%%%%%
%Professional Projects
%%%%%%%%%%%%%%%%%%%%%%%%%%%%%%%%%%%%%%%%%%%%%%%%%%%%%%%%%%%%%%%%%%%%%%%%%%%%%%%%%%%%%
%%%%%%%%%%%%%%%%%%%%%%%%%%%%%%%%%%%%%%%%%%%%%%%%%%%%%%%%%%%%%%%%%%%%%%%%%%%%%%%%%%%%%
\section{Work History}

\affiliation[]{Teamsense}{January 2022 - June 2022}
\begin{compactitem}
\item Contribute to implementation design of new features or architecture as well as implement new features, maintain existing code, and correct defects
\item Take part in rotation to act as support engineer for the customer success team to triage bugs, oversee their resolution and ensure that defects are corrected in accordance with the contract with the client
\item Tested and compared new hardware options for devs to be able to work more efficiently with fewer restrictions due to compute power
\item Researched various dashboard software options, contacted representatives from the ones most closely suiting company needs and presented the information to the team
\item Tech Stack: Python, Django, Typescript, React, Datadog, git
\end{compactitem}

\affiliation[]{Rivers Agile}{April 2019 - January 2022}
\begin{compactitem}
\item Converted existing code to VueJS in order to add internationalization features to prepare for use in Europe
\item Chased bugs, created features, and developed prototypes for real time dashboards and performance reports to showcase the efficiency of the autonomous forklifts to the clients with the goal of aiding in sales
\item Participated in planning meetings for new software projects in which there was integration with other teams in order to fully understand customer needs from the product and then created user stories from that perspective
\item Participated in architectural planning meetings for new projects and tested different technologies by creating rapid prototypes and discussing the pros and cons of different approaches
\item Tech Stack: Python, Plot.ly, Django, RabbitMQ, MQTT, VueJS, node, selenium, i18n, Docker, Vagrant, Gitea
  % websockets
\end{compactitem}

\affiliation[]{Contractor}{February 2018 - Jauary 2019}
\begin{compactitem}
\item Collected data from the FAA pertaining to yearly flight hours per aircraft model and from the NTSB pertaining to aircraft accident data by creating working relationships with officials from both organizations
\item Ran statistical analyses on data to define a standard metric to compare the safety of different aircraft types, determined accidents per 100,000 hours of flight was most accurate and relevant; calculated this for overall safety as well as in differing environmental conditions
\item Created and deployed a website in Golang to display the metrics, which were graphed using Matplotlib
\item Recorded detailed methods regarding the statistical analysis to ensure reproducibility and accountability
\item Created a bot to scrape the NTSB database and alert subscribed twitter users to when a final report was released
\item Tech stack: Python, Jupyter Notebooks, Golang, SQL, Matplotlib
\end{compactitem}

\affiliation[]{Nightingale Security}{October 2017 - February 2018}
\begin{compactitem}
\item Used the bug tracker to locate and correct various bugs in the drone control UI, such as display and logic errors
\item Created last minute fixes for UX issues in time for product demonstrations to clients and venture capitalists
\item Designed a feature to allow the user to create bounding boxes on images which were to be fed into a machine learning algorithm which woudl then be utilized by the drone to automatically identify risks
\item Utilized agile development methods in a remote setting, including Kanban, Slack, and video conferencing
\item Tech stack: JavaScript, PHP, Angular, GoogleMapsAPI, git
\end{compactitem}

\affiliation[]{Topographical Data Analysis}{January - April 2016}
\begin{compactitem}
\item Reverse engineered the recording of data from the Tripod Data Systems Survey Pro in order to correct errors in manual entry of information into the total station and prevent the loss of a day's measurements
\item Utilized the data from the total station to make accurate 3D images of the topography and identify systemic issues in our mapping policies including spacing and user carelessness
\item Tech stack: QGIS, ROOT (CERN), Python, LaTeX
\end{compactitem}

%%%%%%%%%%%%%%%%%%%%%%%%%%%%%%%%%%%%%%%%%%%%%%%%%%%%%%%%%%%%%%%%%%%%%%%%%%%%%%%%%%%%%%%%%%%%%%%%%%%%%%%%%%%%%%%%%%%%%%%%%%%%%%%
% Education
%%%%%%%%%%%%%%%%%%%%%%%%%%%%%%%%%%%%%%%%%%%%%%%%%%%%%%%%%%%%%%%%%%%%%%%%%%%%%%%%%%%%%%%%%%%%%%%%%%%%%%%%%%%%%%%%%%%%%%%%%%%%%%%
\section{Education}

\affiliation []{Fourth Brain}{2020}
\begin{compactitem}[\null]
\item Machine Learning
\end{compactitem}

\affiliation[]{LambdaSchool}{2017-2018}
\begin{compactitem}[\null]
\item Computer Science
\end{compactitem}
% \begin{compactitem}
% \item Completed the full time 6 month computer science course through Lambda School
% \item Learned data structures, algorithms, front end, back end, middleware, and databases
% \item Tech stack: JavaScript, React, Express, Node.js, MongoDB, Docker, Kubernetes
% \item Completed an additional 5 week full time course specializing in Java backend
% \item Tech stack: Java, Spring, Maven, RabbitMQ, SQLite, Swagger
% \end{compactitem}

\affiliation []{Mercyhurst University}{}
\begin{compactitem}[\null]
\item Bachelors of Science in Anthropology
\end{compactitem}


\end{document}
